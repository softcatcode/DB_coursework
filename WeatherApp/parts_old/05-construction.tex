\chapter{Конструкторская часть}

В данном разделе будут представлены схемы алгоритмов работы каждого потока обработки данных.
Они будут конструкторской документацией к разрабатываемой программе.

\section{Схемы алгоритмов}
На рисунке~\ref{fig:process} представлена схема
алгоритма работы одного потока.
Она используется для каждого этапа и иллюстрирует метод взаимодействия, основанный на очередях.
На рисунках~\ref{fig:read}--\ref{fig:save} изображены схемы алгоритмов каждой стадии обработки данных.

\begin{figure}[H]
	\centering
	\includegraphics[height=0.6\textheight, width=\textwidth]{img/process.pdf}
	\caption{
        Схема алгоритма работы одного потока обработки данных
    }
	\label{fig:process}
\end{figure}

\begin{figure}[H]
	\centering
	\includegraphics[height=0.6\textheight, width=\textwidth]{img/read.pdf}
	\caption{
        Схема алгоритма чтения данных
    }
	\label{fig:read}
\end{figure}

\begin{figure}[H]
	\centering
	\includegraphics[height=0.35\textheight]{img/extract.pdf}
	\caption{
        Схема алгоритма извлечения данных
    }
	\label{fig:extract}
\end{figure}

\begin{figure}[H]
	\centering
	\includegraphics[height=0.35\textheight]{img/save.pdf}
	\caption{
        Схема алгоритма сохранения данных
    }
	\label{fig:save}
\end{figure}

\section*{Выводы из конструкторской части}
В данном разделе были разработаны схемы алгоритмов обработки данных на каждой стадии и алгоритма организации их взаимодействия.
Таким образом, получена конструкторская документация к разрабатываемой программе.