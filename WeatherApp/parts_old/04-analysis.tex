\chapter{Аналитическая часть}

\section{Чтение данных}
На первой стадии обработки задачи необходимо прочитать ссылки на рецепты, загруженные в файлы.
Есть следующие категории:
\begin{itemize}
    \item десерты;
    \item закуски;
    \item заготовки;
    \item мясо;
    \item напитки;
    \item мультиварка;
    \item первые блюда;
    \item рыба;
    \item салаты;
    \item выпечка;
    \item гарниры.
\end{itemize}
Для каждой из них хранится папка, содержащая список веб-страниц, в которых следует найти ссылки на рецепты данной категории.
Ссылка на рецепт имеет в html-файле имеет формат: <<href="/node/n">>, где $n$ -- идентификатор рецепта.
Полная ссылка на рецепт имеет вид <<https://2eda.ru/node/n>>.
Таким образом, задача по поиску ссылок сводится к поиску подстроки в строке, содержащейся в html-файле.
В результате первой стадии выполнения задачи получается список ссылок на страницы, содержащие полное описание рецепта.

\section{Извлечение данных}
Вторая стадия обработки заключается в формировании структуры данных рецепта из ссылки на страницу, содержащую его.
Требуется скачать html-файл, содержащий данные рецепта, а затем извлечь эти данные из файла.
Чтение рецепта из html-файла сводится к поиску строковых значений полей его структуры данных.
Пусть $v$ -- значение свойства $p$ в каком-либо рецепте.
Тогда необходимо найти подстроку вида
<meta property="og:p" content="v" />.
Таким образом, можно извлечь значение $v$ для следующих свойств:
\begin{itemize}
    \item заголовок;
    \item ссылка на рецепт;
    \item ссылка на картинку;
    \item описание к рецепту.
\end{itemize}
Идентификатор рецепта получается из ссылки на него -- достаточно перевести в число суффикс после последнего символа <</>>.
Пусть $descr$ -- описание шага приготовления, а $url$ -- ссылка на его изображение.
Тогда следует найти в html-файле подстроку вида
<< <a itemprop="itemListElement"...href="link" title="descr" >>.
Из неё получаются значения $descr$ и $link$.
Для получения описания ингредиента следует найти все подстроки вида <div itemprop="recipeIngredient"...>ingr</div>, где $ingr$ -- строка, описывающая ингредиент.
Она имеет вид <<name -- k unit>>, где $name$ -- название ингредиента, $k$ -- его количество, а $unit$ -- единица измерения.
Таким образом, описаны методы извлечения всех данных, необходимых для формирования рецепта.

\section{Сохранение данных}
На последней стадии обработки задачи извлечённые рецепты сохраняются в базу данных.
Причём структура данных рецепта хранится в виде таблицы.
Таким образом, каждый рецепт будет представлять её строку.
Далее представлены составляющие рецепта и типы данных, описывающих их:
\begin{itemize}
    \item ссылка на рецепт -- строка;
    \item идентификатор -- целое число;
    \item идентификатор задания -- целое число;
    \item ссылка на изображение -- строка;
    \item описание -- строка;
    \item заголовок -- строка;
    \item описания шагов приготовления -- массив строк;
    \item ссылки на изображения шагов приготовления -- массив строк;
    \item ингредиенты -- массив строк;
    \item скачанное изображение -- массив целых чисел.
\end{itemize} 
Для того, чтобы записать список рецептов в базу данных, достаточно подключиться к ней и последовательно добавить соответствующие строки к таблице рецептов.
В результате задача по сохранению рецептов будет выполнена.

\section{Взаимодействие потоков}
В программе должно быть 3 потока, каждый из которых решает свою задачу: чтение, извлечение или сохранение данных.
Поток должен передавать потоку следующей стадии обработки рецепта информацию о новой задаче.
Это делается с помощью очередей.
Каждая стадия обработки рецепта имеет свою очередь задач.
В начале выполнения задача извлекается из очереди, а после выполнения формируется задача для следующей стадии, которая помещается в следующую очередь.
Возникает проблема: 2 потока могут начать выполнять операцию над одной и той же очередью задач одновременно, что приведёт к неопределённому поведению программы.
Поэтому для каждой очереди необходимо использовать средство взаимоисключения -- мьютекс.

\section*{Выводы из аналитической части}
В данном разделе были описаны методы обработки данных на каждой стадии.
Также был описан способ взаимодействия потоков, выполняющих обработку рецептов.