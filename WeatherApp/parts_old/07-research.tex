\chapter{Описание исследования}

В данном разделе будет проведено вычисление среднего времени существования, ожидания и выполнения задач на каждом этапе обработки данных.


\section{Технические характеристики} {
    Технические характеристики устройства, на котором выполнялись замеры по времени, представлены далее:
    \begin{itemize}
        \item процессор: Apple(M2) 2.42 -- 3.48 ГГц;
        \item оперативная память: 8 Гбайт;
        \item операционная система: macOS 14.0 Sonoma.
    \end{itemize}
}

При замерах времени ноутбук был включен в сеть электропитания и был нагружен только системными приложениями.

\section{Вычисление временных характеристик работы \newline программы}
Необходимо рассчитать среднее время существования, ожидания и обработки задач на каждой стадии обработки данных.
Для этого были использованы замеры времён создания, планирования на выполнение, начала обработки и завершения выполнения каждой задачи.
Для каждой стадии обработки рецептов использовался ровно $1$ поток.
Причём создание задач на чтение данных происходило тоже в отдельном потоке.
В результате построены диаграммы для каждой стадии обработки.
Они представлены на рисунках~\ref{fig:read}--\ref{fig:save}

\begin{figure}[H]
	\centering
	\includegraphics[width=0.9\textwidth]{img/read_t.png}
	\caption{Среднее время существования, ожидания и обработки задачи на чтение данных}
	\label{fig:read}
\end{figure}

\begin{figure}[H]
	\centering
	\includegraphics[width=0.9\textwidth]{img/extr_t.png}
	\caption{Среднее время существования, ожидания и обработки задачи на извлечение данных}
	\label{fig:extr}
\end{figure}

\begin{figure}[H]
	\centering
	\includegraphics[width=0.9\textwidth]{img/save_t.png}
	\caption{Среднее время существования, ожидания и обработки задачи на сохранение данных}
	\label{fig:save}
\end{figure}

Из диаграмм следует, что задача в среднем ждёт начала своего выполнения большую часть времени существования.
Причём отношение времени обработки к времени ожидания меньше всего на стадии извлечения.
Это объясняется тем, что вторая стадия обработки рецептов -- самая трудоёмкая, следовательно, задачи на извлечение данных будут долго ожидать своей очереди.

\section{Выводы из исследовательской части}
В данном разделе было вычислено среднее время существования, ожидания  обработки каждой задачи для каждой стадии обработки данных.
Также были приведены соответствующие диаграммы.