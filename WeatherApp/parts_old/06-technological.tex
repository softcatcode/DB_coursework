\chapter{Технологическая часть}

В данном разделе описаны средства реализации программы, а также будет приведена её реализация.

\section{Средства реализации}
Для реализации данной лабораторной работы был выбран язык \texttt{C++}.
В нём есть библиотека <<pqxx>> для подключения к базе данных~[1].
Cтандартная библиотека \texttt{ctime} позволяет выполнить замеры времени работы программы~[4].
Создание потока обеспечивается функцией pthread\_create~[2], а ожидание его завершения -- функцией pthread\_join~[3].
Также данный язык предоставляет библиотеку \textttt{std::mutex}, содержащую средство взаимоисключения -- мьютекс~[5].

\section{Сведения о модулях программы}
Разработанная программа содержит следующие модули:
\begin{itemize}
	\item $main.cpp$ -- точка входа и логика взаимодействия с пользователем;
	\item $task\_executor.cpp$ -- реализация обработки данных;
	\item $task\_creator.cpp$ -- реализация создания, планирования и логирования задач;
	\item $food\_entities.cpp$ -- структуры данных рецепта, ингредиента и шага рецепта.
\end{itemize}

\section{Реализация алгоритмов}

Листинг~\ref{lst:process} содержит реализацию каждого процесса обработки данных и их запуск на выполнение, а листинги~\ref{lst:read}--\ref{lst:save} -- реализацию чтения, извлечения и сохранения данных.
Реализация создания задач для стадии чтения представлена в листинге~\ref{lst:create}.

\lstinputlisting[
    label = lst:process,
    caption= Реализация функций-процессов обработки данных и их запуска на выполнение
] {code/process.cpp}

\lstinputlisting[
    label = lst:create,
    caption=Реализация создания задач для стадии чтения
] {code/create.cpp}

\lstinputlisting[
    label = lst:read,
    caption = Реализация выполнения стадии чтения данных
] {code/read.cpp}

\lstinputlisting[
    label = lst:extract,
    caption = Реализация выполнения стадии извлечения данных
] {code/extract.cpp}

\lstinputlisting[
    label = lst:save,
    caption = Реализация выполнения стадии сохранения данных
] {code/save.cpp}

\section*{Выводы из технологической части}
В данном разделе были реализована программа для конвейерной обработки рецептов, загруженных с сайта.
Также была приведена информации о выбранных средствах разработки.