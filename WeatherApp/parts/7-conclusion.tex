\chapter*{ЗАКЛЮЧЕНИЕ}
\addcontentsline{toc}{chapter}{ЗАКЛЮЧЕНИЕ}

В результате выполнения данной работы были выполнены следующие задачи:
\begin{itemize}
    \item представлен список параметров погоды вместе с их граничными значениями;
    \item сформулированы требования к базе данных;
    \item спроектированы сущности базы данных и выбран способ их хранения;
    \item спроектирована ролевая модель базы данных;
    \item спроектирована архитектура приложения;
    \item выбраны средства реализации базы данных и приложения;
    \item приложение, использующее разработанную базу данных, протестировано;
    \item исследовано влияние индекса на временные характеристики работы базы данных.
\end{itemize}

В результате исследования определено, что индексы ускоряют прецедент обновления массива стран на $67\%$, но замедляют вставку массива новых стран на $20\%$.
На прецеденты, связанные с погодой индексы не оказывают существенного влияния.
Также выяснено, что индекс для таблицы городами не нужен, так как они извлекаются из базы данных только по первичному ключу.

Таким образом, все задачи работы выполнены и цель достигнута.