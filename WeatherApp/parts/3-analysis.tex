\chapter{Аналитическая часть}

\section{Анализ предметной области}

Техническое задание к данной работе предполагает создание базы данных для погодного приложения.
Следовательно, необходимо проанализировать предметную область метеорологии для выделения сущностей и связей между ними, а также атрибутов данных сущностей.

\subsection{Cущности в метеорологии}
Главной задачей метеорологии является прогнозирование погоды~[2].
Тогда основной сущностью исследуемой предметной области является погода.
Она определяется для какого-либо дня или часа.
Причём день представляется датой, а час -- временем относительно начала соответствующего дня.
Следовательно, день и час являются атрибутами погоды, а не отдельными сущностями.
Если погода определена для будущего времени, то она является прогнозом.
Поэтому нет смысла выделять прогноз как отдельную сущность.

Важной особенностью погоды является локальность.
Это значит, что она всегда ассоциируется с определённым географическим местом.
В сервисах, предоставляющих информацию о погоде, в качестве такого места рассматривается город~[4]~[5].
Тогда город -- следующая сущность предметной области.

Таким образом, к сущностям предметной области метеорологии относятся погода и город.

Далее рассмотрены составляющие погоды, представляющие интерес для пользователей её прогнозов~[1]. 
Их список:
\begin{itemize}
    \item температура воздуха;
    \item атмосферное давление;
    \item скорость ветра;
    \item относительная влажность воздуха;
    \item видимость;
    \item облачность;
    \item осадки;
    \item гроза;
    \item туман;
    \item метель;
    \item пыльные бури;
    \item гололёд.
\end{itemize}
Ещё один часто выделяемый параметр погоды -- ощущаемая температура~[4]~[5].
Он нужен для субъективного понимания пользователем текущей погоды.
Перечисленные параметры, в зависимости от способа формализации, могут быть как числовыми, так и логическими, и являются атрибутами погоды.

\subsection{Формализация сущностей предметной области}
В данном разделе будут формализованы сущности погоды и города.
Для формализации понятия погоды необходимо выделить набор формальных параметров погоды на основе выделенных её составляющих.
Для этого проанализированы существующие погодные сервисы: Яндекс Погода~[4], AccuWeather~[5], Погода <<Mail.ru>>~[6], The Weather Channel~[7].
Такие параметры погоды, как температура, давление, влажность и скорость ветра являются числовыми и измеряются в градусах по Цельсию, миллиметрах ртутного столба, процентах и метрах в секунду соответственно.
Видимость характеризуется максимальным расстоянием, на котором человек может что-либо увидеть и измеряется в километрах.
Погодные сервисы следят за двумя видами осадков: дождевыми и снежными и измеряют их в миллиметрах.
Таким образом, осадки формально описываются двумя числами: количеством выпавшего дождя и снега в миллиметрах.
Далее рассмотрены параметры погоды, не поддающиеся численному описанию.
К ним относится облачность и наличие каких-либо особых природных явлений.
Облачность представляется перечислением и чаще всего имеет следующие виды: солнечно, облачно с прояснениями, пасмурно.
Природные явления также задаются перечислением: гроза, дождь, туман, солнце, снег и метель.
Таким образом, удобно ввести тип погоды -- перечисление, которое будет содержать информацию как об облачности, так и о природных явлениях.
Оно будет включать следующие значения:
\begin{itemize}
    \item гроза;
    \item морось;
    \item дождь;
    \item ливень;
    \item туман;
    \item солнце;
    \item облачно с прояснениями;
    \item пасмурно;
    \item снег;
    \item метель.
\end{itemize}
Таким образом, пользователь сможет легко интерпретировать погоду.

Необходимо также выделить параметры такой сущности как город.
Он характеризуется названием, географическим расположением и названием страны.
Под географическим расположением подразумевается широта и долгота, выраженные в вещественных числах~[3].

Далее рассмотрены сущности, относящиеся ко времени.
День описывается датой, которая содержит год, месяц и день.
Час в прогнозе почасовой погоды всегда привязан к определённому дню и указывается порядковым номером относительно его начала.

Таким образом, все сущности предметной области метеорологии выделены и формализованы.

\subsection{Граничные значения параметров сущностей}
После описания сущностей предметной области необходимо представить требования к их целостности.
Далее представлены таблицы~\ref{table:weather_parameters}--\ref{table:city_parameters}, содержащие требования к числовым параметрам погоды и города.
\clearpage
\begin{table}[h!]
    \centering
    \begin{tabular}{ |c|c|c| }
        \hline
            \textbf{Параметр погоды} & \textbf{Ограничения} & \textbf{Единицы измерения} \\
        \hline
            температура & $-273$ -- $\infty$ & \textdegree C \\
        \hline
            ощущаемая температура & $-273$ -- $\infty$ & \textdegree C \\
        \hline
            скорость ветра & $0$ -- $\infty$ & м/с \\
        \hline
            давление & $0$ -- $\infty$ & мм рт. ст. \\
        \hline
            относительная влажность & $0$ -- $100$ & \% \\
        \hline
            дождевые осадки & $0$ -- $\infty$ & мм \\
        \hline
            снежные осадки & $0$ -- $\infty$ & мм \\
        \hline
            время восхода & $00:00$ -- $23:59$ & 24ч-формат \\
        \hline
            время заката & $00:00$ -- $23:59$ & 24ч-формат \\
        \hline
    \end{tabular}
    \caption{\centering Параметры погоды, их единицы измерения и граничные значения}
    \label{table:weather_parameters}
\end{table}

\begin{table}[h!]
    \centering
    \begin{tabular}{ |c|c|c| }
        \hline
            \textbf{Параметр города} & \textbf{Ограничения} & \textbf{Единицы измерения} \\
        \hline
            название & от 1 символа & --- \\
        \hline
            широта & $-90$ -- $90$ & градусы \\
        \hline
            долгота & $-180$ -- $180$ & градусы \\
        \hline
    \end{tabular}
    \caption{\centering Параметры города, их единицы измерения и граничные значения}
    \label{table:city_parameters}
\end{table}

\section{Требования к базе данных}
Для проектирования базы данных необходимо определить набор требований к ней.
К ним относится, во-первых, хранимые отношения, а во-вторых, функции, предоставляемые базой данных.
Далее представлен список пользовательских возможностей, предоставляемых базой данных:
\begin{itemize}
    \item получить координаты точек графика изменения погодного параметров;
    \item получить список заметок, редактируемых пользователем;
    \item сохранение и удаление заметки;
    \item получить список городов, начинающихся с заданной подстроки;
    \item получить прогноз погоды на $n$ следующих дней;
    \item добавление и удаление города из избранного;
    \item получить список избранных городов пользователя;
    \item получить историю погоды для определённого периода времени;
    \item получить или обновить иконку погоды по её идентификатору;
    \item получить список дней с погодой, соответствующей заданным параметрам.
    \item получить почасовую погоду для текущего дня в выбранном городе;
    \item удалить погодные данные для определённого периода времени;
    \item дать пользователю права на редактирование заметки;
    \item .
\end{itemize}
Исходя из функций базы данных, она должна иметь следующие сущности:
\begin{itemize}
    \item погода;
    \item город;
    \item страна;
    \item график;
    \item заметка;
    \item пользователь;
    \item иконка погоды;
    \item погодный параметр.
\end{itemize}
Также необходимо обеспечить хранение двух отношений многие ко многим:
\begin{itemize}
    \item пользователь -- избранный им город;
    \item пользователь -- редактируемая им заметка.
\end{itemize}
Для этого нужно ввести специальные сущности связи между пользователями и городами и заметками.
Таким образом, для каждого пользователя можно будет загрузить как список городов, которые он добавил в избранное, так и список заметок, которые он может редактировать.
Важным замечанием является то, для обеспечения переноса данных с одного устройства на другое необходимо хранить определённые сущности в отдельной базе данных, к которой могут подключиться все пользователи.
К таким сущностям относятся пользователь, заметка, а также сущности связи.

Для выполнения предъявленных требований достаточно реляционной базы данных.
Этот тип баз данных основан на хранении информации в виде таблиц и является самым популярным для решения практических задач~[8].

\section*{Выводы из аналитической части}
В данном разделе была проанализирована предметная область метеорологии, в результате чего определены её основные сущности, а также атрибуты погоды и их граничные значения.
Затем определены данные, хранимые в базе данных, и функции, которые она должна предоставлять.