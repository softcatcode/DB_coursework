\chapter*{ВВЕДЕНИЕ}
\addcontentsline{toc}{chapter}{ВВЕДЕНИЕ}

\textbf{Целью} данной работы, согласно техническому заданию, является разработка базы данных для хранения и обработки данных погодного приложения.

Для её достижения необходимо решить следующие \textbf{задачи}:
\begin{itemize}
    \item определить список возможных параметров погоды и их граничные значения;
    \item сформулировать требования к базе данных;
    \item спроектировать сущности базы данных и способ их хранения;
    \item спроектировать ролевую модель базы данных;
    \item спроектировать архитектуру приложения;
    \item выбрать средства реализации базы данных и приложения;
    \item протестировать приложение, использующее разработанную базу данных;
    \item исследовать влияние индекса на временные характеристики работы базы данных.
\end{itemize}

\section*{Актуальность}
Далее дано обоснование актуальности работы.
Существуют сервисы, собирающие данные о погоде с большинства мировых метеорологических станций~[19].
Но чтобы эти данные могли быть использованы конечными пользователями, необходимы специальные приложения.
Они скачивают с сервера только ту информацию о погоде, которая нужна конкретному пользователю и отображают её в удобном для него формате.
Более того, скачанные данные удобно сохранить на устройство для того, чтобы осуществлять доступ к ним даже при отсутствии соединения с сервером.
Именно эту функцию будет предоставлять разрабатываемая база данных.
Это сделает использование погодного приложения более удобным.
Таким образом, актуальность разработки базы данных обоснована.
